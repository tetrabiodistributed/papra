%% Hardware Metapaper Template for the Journal of Open Hardware, v.1

\documentclass[a4paper]{article}

\usepackage{lmodern}
\usepackage[T1]{fontenc}
\usepackage[utf8]{inputenc}
\usepackage{amssymb,amsmath}

\usepackage{hyperref}
\hypersetup{unicode=true,
            pdfborder={0 0 0},
            breaklinks=true}
\urlstyle{same}

\usepackage{longtable,booktabs}
\IfFileExists{parskip.sty}{%
\usepackage{parskip}
}{% else
\setlength{\parindent}{0pt}
\setlength{\parskip}{6pt plus 2pt minus 1pt}
}
\setlength{\emergencystretch}{3em}  % prevent overfull lines
\providecommand{\tightlist}{%
  \setlength{\itemsep}{0pt}\setlength{\parskip}{0pt}}
\setcounter{secnumdepth}{0}

%% \date{}


\title{Template for Hardware Metapaper}
\usepackage{authblk} 
\author[1]{Author 1}
\author[2]{Author 2} 
\affil[1]{Author 1, Open Hardware project affiliation} 
\affil[2]{Author 2, institutional affiliation}
\renewcommand\Affilfont{\itshape\small}

\begin{document}
\maketitle


% Include your abstract here
\begin{abstract}
A short, 300 words summary of the hardware being described: what problem(s) the hardware addresses, what it does, how it technically/methodologically advances the state-of-the-art, how it was designed and implemented, and its applicability to other issues/research/areas of reuse.
\end{abstract}

\begin{longtable}[]{@{}l@{}}
\begin{minipage}[t]{0.97\columnwidth}\raggedright\strut


\subsection{Metadata Overview}\label{h.akaipbqoqfs8}

Main design files: link to repository with design files and assembly instructions

Target group: scientists in biology, engineers for example

Skills required: 3D printing - easy; CNC milling of aluminum - advanced;

Replication\textsuperscript{\protect\hyperlink{cmnt1}{{[}a{]}}}{:
}{www.irnas.eu for instance.}

See section ``Build Details'' for more detail.


\subsection{Keywords}\label{h.kdz351yp7g7c}

{(required)}{~bioprinting; open source; extruder;}

\strut\end{minipage}\tabularnewline
\bottomrule
\end{longtable}


\subsection{Introduction}\label{h.pnj38xyr5dyy}

{Bioprinting is a fast growing field, APPLICATION, TECHNOLOGY, why OS...}


\subsection{Overall Implementation and design}\label{h.1u7vph94gfbt}

The hardware of Vitaprint is divided into two parts, namely the CNC part and the extruder. The CNC construction the miniCNC design by AL was used. MiniCNC is a small and robust version of a standard 3-axis CNC router. The support is made of steel and holds in place the aluminium head and table. Stainless steel rods are used with ball bearings to ensure smooth movement. Linear motion is controlled by three Nema32 stepper motors coupled with ball screws.

The second part of the Vitaprint hardware is the syringe extruder, designed by LB. The extruder is entirely built out of aluminium. Most of the parts are milled out of 5mm aluminium plate to keep the manufacturing easy. Two parts are manufactured out of thicker aluminium and require more advanced CNC milling skills.

The extruder is designed to fit the outer dimensions of standard 5ml syringes. The syringe is placed firmly inside the extruder base and fixed with a front cap without any screws. A narrow strip of EPDM is used as a spring to keep the syringe in place. The extruder head has milled holes for standard heaters that are used in hot ends of other 3D printers. Nema14 non-captive linear stepper motor is used to move the syringe piston.

A common feature of most high-end 3D bioprinters is multiple-extruder technology and fast extruder switching. A 24BYJ-48 stepper motor was therefore added to each extruder to add the possibility to move each extruder in the Z-direction separately.


\subsubsection{Universality of the design}\label{h.q32f2nclh4e5}

One of the biggest advantages that Vitaprint brings is that each extruder is a self-contained unit. First and foremost, the user is able to add as many extruders as needed to the head without significant modifications of any of the hardware pieces. Moreover, the extruder unit can be attached to any other CNC system with very little modification. This means that the user can build a Vitaprint extruder unit for their own CNC router.


\section{(2) Quality control}\label{h.f8237gmzmwc6}

\subsection{Safety}\label{h.v60aduckfisj}

Describe all relevant safety issues or reference to a risk assessment
if included in the hardware documentation. Detail what safety
considerations have been included in the hardware design and how these
features have been tested, including any official safety standards or
criteria that were used in this assessment. If appropriate, discuss the
wider context of use of the hardware and safety issues or risks that may
arise in the use environment.


\subsection{Calibration}\label{h.kr90wh14sxr5}

If the hardware is used for measurements, please detail here how the
reliability of measurements, or other hardware properties that are
relevant for measurements, has been quantified and explain the
results. Be clear about the processes or procedures used to compare the
hardware to a standard, as well as the description of the standard
calibrated against.

Detail the general procedures in place for users to calibrate their
hardware before or during use. What methods can be used to relate user
generated data to data from other sources? 

Note: Detailed instructions belong in documentation; here, provide
insight into how and why the calibration is valid.


\subsubsection{Subsections}\label{h.6mrkl1u5j8xc}

We encourage the use of subsections within all sections to increase
clarity.


\subsection{General testing}\label{h.wbekh9ay82yu}

In this section, details can be provided on the testing of hardware
functionalities, that are not directly essential for precision
operation of the hardware in the given context (which are in turn,
where applicable, handled under Calibration), such as automated
movements to position the hardware, repeatability of tool exchanges,
recyclability, water-tightness, weight or other possibly relevant
characteristics. We encourage the authors to characterise all
appropriate functionalities of the hardware, if not already described
elsewhere (add reference instead). The testing should define the
safe/reliable limits in which the components can be operated (e.g. step
size and repeatability of linear motion, force ranges, ratio of devices
with leaks when built in a workshop, etc).This will enhance the
usability of the hardware or method in other contexts.

Again: Detailed instructions belong in documentation; here, provide a
summary ~instead.


\section{(3) Application}\label{h.f78bi3oom0mu}

\subsection{Use case(s)}\label{h.4q5g9edishy3}

Describe at least one example of an application of your hardware. This
should include some evidence of output, e.g. data produced by the use of
the device or a picture of other types of results. Outline how the
quality control in the previous section enables the use of the hardware
in this context. We encourage the inclusion of experiment results or the
reference to a publication (published or to-be-published) where these
results are detailed. We also encourage pointers to ongoing work.

Note: In the spirit of openness, we require authors to provide (or link
to) datasets along with the submitted graphic representations. We do not
impose arbitrary limits on inclusion of data so please include
sufficient empirical detail and results to ensure your data can be
easily verified, analysed and clearly interpreted by the wider
scientific community.


\subsubsection{Subsections}\label{h.qz4dez1pbkv1}

We encourage the demonstration of different use cases, divided by
sub-sections to guide the reader.


\subsection{Reuse potential and adaptability}\label{h.6wkumyl0ejrh}

Please describe in as much detail as possible the ways in which the
hardware could be reused by other researchers both within and outside of
your field. This should include the use cases for the hardware, and also
details of how the hardware might be modified or extended (including how
contributors should contact you) if appropriate.Refer to section
``Ease of build'' where necessary.

Please provide your thoughts on the adaptability of the hardware
design. What tools and procedures would it require for other users to
modify your design without your help in order to adapt them for
foreseeable or even unforeseeable use.

Also you must include details of what support mechanisms there are in
place for this hardware and software (even if there is no support or
support community).


\section{(4) Build Details}\label{h.l8i9vokvs0bj}

\subsection{Availability of materials and methods}\label{h.60suejv0jlzi}

Summarise what materials have been used to construct the hardware and
what methods to process the materials as well as the assembly. Provide
more details or references where important materials or methods are
non-standard, not globally available or produced only by one
manufacturer.


\subsection{Ease of build}\label{h.wg823sgyb1e4}

Have any measures been taken in the design to make the hardware easy to
build for other users e.g. reduction of parts, features in the design
to make the hardware assembly more reliable?


\subsection{Operating software and peripherals}\label{h.uz77dixfh5i4}

If hardware requires software, details on the operating software and
programming language - Please include minimum version compatibility.
Additional system requirements, e.g. memory, disk space, processor,
input or output devices.

If the hardware does not require software, detail any required
supporting processes or protocols required for use.


\subsection{Dependencies}\label{h.vr0vnjs8z9ar}

E.g. other hardware or software projects, modular components,
libraries, frameworks, incl. minimum version compatibility.


\subsection{Hardware documentation and files location:}\label{h.nbisrsde6sc3}

Archive for hardware documentation and build files (required.
We recommend the use the DocuBricks repository, please see author guide for criteria and alternative
repositories.) Note: We require the inclusion of modifiable design
files as well as a detailed documentation of the functionality of the
hardware with assembly instructions. This will be assessed as part of
the journal peer review process.

Name: The name of the archive

Persistent identifier: e.g. DOI, etc.

Licence: Open hardware license under which the documentation and
files are licensed - see author guide for more information

Publisher: Name of the person who deposited the documentation

Date published: dd/mm/yy

Modifiable design files (if different from above)

Name:The name of the emulation environment

Persistent identifier: e.g. DOI, handle, PURL, etc.

Licence: Open license under which the software is licensed here

Publisher: Name of the person who deposited the documentation

Date published: dd/mm/yy

Software code repository (e.g. SourceForge, GitHub etc.)
(required)

Name: The name of the code repository

Identifier: {The identifier (or URI) used by the repository

Licence: Open license under which the software is licensed

Date published: dd/mm/yy


\section{(5) Discussion}\label{h.90jl7wm65t65}

\subsection{Conclusions}\label{h.h3fr33ylzsnh}

Conclusions, learned lessons from design iterations, learned lessons
from use cases, summary of results.


\subsection{Future Work}\label{h.neocsr410zj}

Further work pursued by the authors or collaborators; known issues;
suggestions for others to improve on the hardware design or testing,,
given what you have learned from your design iterations.


\subsection{Paper author contributions}\label{h.fy8hbipy6kwe}

Task (e.g. design, assembly, use cases contribution, documentation,
paper writing), contribution, author name.


\subsection{Acknowledgements}\label{h.gu3yyarx72d6}

Please add any relevant acknowledgements to anyone else who supported
the project in which the hardware was created, but did not work directly
on the hardware itself.

Please list anyone who helped to create the hardware and software (who
may also not be an author of this paper), including their roles and
affiliations.


\subsection{Funding statement}\label{h.4u1a7tugh2om}

If the hardware resulted from funded research please give the funder
and grant number.


\subsection{Competing interests}\label{h.q1j1rznb43fl}

If any of the authors have any competing
interests then these must be declared. The authors' initials should be used to denote
differing competing interests. For example: ``BH has minority shares in
{[}company name{]}, which part funded the research grant for this
project. All other authors have no competing interests.''

If there are no competing interests, please add the statement:

``The authors declare that they have no competing interests.''


\subsection{References}\label{h.6fml9tf50r5c}

Please enter references in the Harvard style and include a DOI where
available, citing them in the text with author and year.


\section{Copyright notice}\label{h.jm5gcqv4g8x0}

Authors who publish with this journal agree to the following terms:

Authors retain copyright and grant the journal right of first
publication with the work simultaneously licensed under
a Creative Commons Attribution License that allows others to share the work with
an acknowledgement of the work's authorship and initial publication in
this journal.

Authors are able to enter into separate, additional contractual
arrangements for the non-exclusive distribution of the journal's
published version of the work (e.g., post it to an institutional
repository or publish it in a book), with an acknowledgement of its
initial publication in this journal.

By submitting this paper you agree to the terms of this Copyright
Notice, which will apply to this submission if and when it is published
by this journal.


\end{document}